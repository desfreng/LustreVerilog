\documentclass{beamer}
\usepackage[T1]{fontenc}
\usepackage[english]{babel}
\usepackage[justification=centering]{caption}
\usepackage{amsmath}
\usepackage{amsfonts}
\usepackage{hyperref}
\usepackage{minted}
\usepackage{ebproof}
\usepackage{mathtools}
\usepackage{tikz}
\usepackage{xcolor}
\usepackage{tcolorbox}

\usetikzlibrary{positioning}

\setbeamertemplate{navigation symbols}{}
\setbeamertemplate{itemize item}[circle]

\usetheme{Singapore}

\title{Lustre \texorpdfstring{$\to$}{->} Verilog}
\author{Gabriel Desfrene}
\date{29 January 2025}


\DeclareMathOperator{\Bool}{Bool}
\DeclareMathOperator{\Int}{Int}
\DeclareMathOperator{\Signed}{Signed}
\DeclareMathOperator{\Unsigned}{Unsigned}
\DeclareMathOperator{\Raw}{Raw}

\begin{document}

\maketitle

\section{Goal}

\subsection{Verilog}

\begin{frame}{The Verilog Language}
    \begin{itemize}
        \item Hardware Description Language (HDL) used to model digital circuits.
        \item  Developed in 1984.
        \item Syntax similar to C.
        \item Supports multiple paradigms: structural \& behavioral.
    \end{itemize}
\end{frame}

\begin{frame}[fragile]{Example 1}
    \begin{minted}{verilog}
        module main (
            input wire clock,
            input wire reset,
            input wire x,
            output reg [7:0] y
        );
          always @(posedge clock) begin
            if (reset) y <= 8'd0;
            else if (x || y > 0) y <= y + 8'd1;
          end
        endmodule
    \end{minted}
\end{frame}

\begin{frame}[fragile]{Example 2}
    \begin{minted}{verilog}
        module main (
            input  wire clock,
            input  wire reset,
            input  wire x,
            output reg  y
        );
          wire new_y, reset_n, x_or_y;

          not (reset_n, reset);
          or (x_or_y, x, y);
          and (new_y, reset_n, x_or_y);

          always @(posedge clock) begin
            y <= new_y;
          end
        endmodule
    \end{minted}
\end{frame}

\begin{frame}{Conclusion}
    \begin{LARGE}
        \begin{center}
            \bf
            It's verbose and it's ugly.
        \end{center}
    \end{LARGE}
\end{frame}

\subsection{Goal}

\begin{frame}{Objectif}
    \begin{LARGE}
        \begin{center}
            \bf
            Compiling Lustre into Verilog.
        \end{center}
    \end{LARGE}
\end{frame}

\begin{frame}{Compiling Lustre into Verilog}
    \begin{itemize}
        \item \textbf{Why?}
              \begin{itemize}
                  \item Take advantage of Lustre's elegance,
                  \item Synthesize Lustre models,
                  \item It looks fun.
              \end{itemize}
              \pause{}
        \item \textbf{Why Verilog?}
              \begin{itemize}
                  \item Standard language for hardware synthesis,
                  \item Allows description in terms of logic gates,
                  \item It's a good opportunity to use it a bit.
              \end{itemize}
    \end{itemize}
\end{frame}

\begin{frame}{Specifications}
    \begin{itemize}
        \setlength{\itemsep}{5pt}
        \item Compile a subset of Lustre into \emph{gate-level} Verilog.
        \item Maintain \emph{retrocompatible} syntax with existing Lustre compilers.
        \item Add operations to handle buses.
    \end{itemize}
\end{frame}

\section{Adapting Lustre}

\begin{frame}{The Lustre Kernel}
    \begin{itemize}
        \setlength{\itemsep}{6pt}
        \item Unary operations:
              \begin{itemize}
                  \setlength{\itemsep}{3pt}
                  \item \texttt{not}: $\Bool \to \Bool$
                  \item \texttt{neg}: $\Int \to \Int$
              \end{itemize}
        \item Binary operations:
              \begin{itemize}
                  \setlength{\itemsep}{3pt}
                  \item \texttt{and}, \texttt{or}: $\Bool \to \Bool \to \Bool$
                  \item $+, -$: $\Int \to \Int \to \Int$
                  \item $=, \neq, <, \leqslant, \geqslant, >$: $\Int \to \Int \to \Bool$
              \end{itemize}
        \item Branching expressions: $\Bool \to \tau \to \tau \to \tau$
        \item \texttt{fby}: $\tau \to \tau \to \tau$
    \end{itemize}
\end{frame}

\begin{frame}{Using Bit-Vectors}
    \begin{itemize}
        \setlength{\itemsep}{6pt}
        \item $\Int \coloneqq \Signed_{\gamma}$
        \item Unary operations:
              \begin{itemize}
                  \setlength{\itemsep}{3pt}
                  \item \texttt{not}: $\Bool \to \Bool$
                  \item \texttt{neg}: $\Signed_{\sigma} \to \Signed_{\sigma}$
              \end{itemize}
        \item Binary operations:
              \begin{itemize}
                  \setlength{\itemsep}{3pt}
                  \item \texttt{and}, \texttt{or}: $\Bool \to \Bool \to \Bool$
                  \item $+$: $\tau \to \tau \to \tau$ \hfill $\tau \in\{\Signed_{\sigma}, \Unsigned_{\sigma}\}$
                  \item $-$: $\Signed_{\sigma} \to \Signed_{\sigma} \to \Signed_{\sigma}$
                  \item $=, \neq, <, \leqslant, \geqslant, >$: $\tau \to \tau \to \Bool$ \hfill $\tau \in\{\Signed_{\sigma}, \Unsigned_{\sigma}\}$
              \end{itemize}
        \item Branching expressions: $\Bool \to \tau \to \tau \to \tau$
        \item \texttt{fby}: $\tau \to \tau \to \tau$
    \end{itemize}
\end{frame}

\begin{frame}{Bit-Vectors Operation}
    \begin{itemize}
        \setlength{\itemsep}{6pt}
        \item \texttt{slice} $i$: $\Raw_{\sigma} \to \Bool$ \hfill $0 \leqslant i < \sigma$
        \item \texttt{select} $[i:j]$: $\Raw_{\sigma} \to \Raw_{j - i}$ \hfill $0 \leqslant i < j \leqslant \sigma$
        \item \texttt{concat}: $\Raw_{\sigma_1}|\Bool \to \Raw_{\sigma_2}|\Bool \to \Raw_{\sigma_1 + \sigma_2}$
        \item Conversions: $\Raw_{\sigma} \to \Signed_{\sigma}, \Signed_{\sigma} \to \Unsigned_{\sigma}, \dots$
        \item $=, \neq$: $\Raw_{\sigma} \to \Raw_{\sigma} \to \Bool$
    \end{itemize}
\end{frame}

\begin{frame}[fragile]{Finally}
    \visible<2->{
        \begin{tikzpicture}[remember picture,overlay]
            \node [rotate=20,scale=5,text opacity=0.3]
            at (current page.center) {MINIJAZZ};
        \end{tikzpicture}
    }

    \begin{minted}{ocaml}
    node after(x, reset: bool) returns (after: bool);
    let
        after = if reset
                then false
                else x or (false fby after);
    tel

    node main(reset, x: bool) returns (y: u8);
    let
        y = 0 fby if after(x, reset) then y + 1 else 0;
    tel
    \end{minted}
    \pause{}
\end{frame}

\section{Implementation}

\subsection{Haskell Propaganda}

\begin{frame}{Using Haskell}
    \begin{columns}
        \column{0.5\textwidth}
        \centering
        \textit{Rust: Borrow checker required}
        \newline
        \textbf{\large Haskell: No need, it's immutable!}

        \vspace{1cm}

        \centering
        \textit{Rust: ``Fearless Concurrency''}
        \newline
        \textbf{\large Haskell: Concurrency is just a Monad!}

        \column{0.5\textwidth}

        \centering
        \textit{Rust: Lifetimes are hard}
        \newline
        \textbf{\large Haskell: Garbage Collector does it for you!}

        \vspace{1cm}

        \centering
        \textit{Rust: Memory safe with effort}
        \newline
        \textbf{\large Haskell: Memory safe by default!}
    \end{columns}

    \pause{}
    \vfill
    \centering
    \textbf{\Large Choose Haskell. Be functional. Be safe.\\Be pure.}
\end{frame}

\subsection{Compiler Stages}

\begin{frame}{Parsing}
    \begin{itemize}
        \setlength{\itemsep}{8pt}
        \item We use \emph{Megaparsec}, a \emph{Monadic Parser} to parse our Lustre grammar,
        \item Monads Used: 2.
    \end{itemize}
\end{frame}

\begin{frame}[fragile]{Parsing Tree}
    \begin{footnotesize}
        \begin{minted}{haskell}
type Expr = Pos ExprDesc
data ExprDesc
  = ConstantExpr Constant
  | IdentExpr (Pos Ident)
  | UnOpExpr UnOp Expr
  | BinOpExpr BinOp Expr Expr
  | ConvertExpr BitVectorKind Expr
  | ConcatExpr Expr Expr
  | SliceExpr Expr (Int, Int)
  | SelectExpr Expr Int
  | AppExpr (Pos Ident) [Expr]
  | TupleExpr (BiList Expr)
  | IfExpr Expr Expr Expr
  | FbyExpr Expr Expr
  deriving (Show, Eq)

type Pattern = Tree (Pos Ident)
data Equation = Equation Pattern Expr
    deriving (Show, Eq)
        \end{minted}
    \end{footnotesize}
\end{frame}

\begin{frame}[fragile]{Typing Tree}
    \begin{footnotesize}
        \begin{minted}{haskell}
data TExpr atyp
  = ConstantTExpr Constant atyp
  | VarTExpr VarId atyp
  | UnOpTExpr UnOp (TExpr atyp) atyp
  | BinOpTExpr BinOp (TExpr atyp) (TExpr atyp) atyp
  | IfTExpr VarId (TExpr atyp) (TExpr atyp) atyp
  | ConcatTExpr (TExpr atyp) (TExpr atyp) atyp
  | SliceTExpr (TExpr atyp) (BVSize, BVSize) atyp
  | SelectTExpr (TExpr atyp) BVSize atyp
  | ConvertTExpr (TExpr atyp) atyp
  deriving (Show)

type TArg = Either (TConstant atyp) VarId
data TEquation atyp
  = SimpleTEq VarId (TExpr atyp)
  | FbyTEq VarId (TExpr atyp) (TExpr atyp)
  | CallTEq (NonEmpty VarId) NodeIdent [TArg]
  deriving (Show)
        \end{minted}
    \end{footnotesize}
\end{frame}

\begin{frame}{Typing}
    \begin{itemize}
        \setlength{\itemsep}{8pt}
        \item Constants are typed dynamically:
              \[10: \Raw_{\geqslant 4}|\Unsigned_{\geqslant 4}|\Signed_{\geqslant 5}\]
        \item Equation normalization on the fly.
        \item Monads Used: 11
              ($+6$ for Typing)
              ($+3$ for Causality)
    \end{itemize}
\end{frame}

\begin{frame}[fragile]{Expression Flattening}
    \begin{small}
        \begin{minted}{haskell}
data CAction
  = SetValCAct CVal
  | UnOpCAct CUnOp CVal
  | BinOpCAct CBinOp CVal CVal
  | IfCAct {ifCond :: VarId, ifTrue :: CVal, ifFalse :: CVal}
  | FbyCAct {initVar :: CVal, nextVar :: CVal}
  | ConcatCAct CVal CVal
  | SliceCAct CVal (BVSize, BVSize)
  | SelectCAct CVal BVSize
  deriving (Show)

data CEquation
  = SimpleCEq CVar CAction
  | CallCEq (NonEmpty CVar) NodeIdent [CVal]
  deriving (Show)
        \end{minted}
    \end{small}

    Monads Used: 12 $(+ 1)$.
\end{frame}

\begin{frame}[fragile]{Verilog Conversion}
    \begin{small}
        \begin{minted}{haskell}
data ModuleInst = ModuleInst
  { name :: ModuleName,
    staticArgs :: [StaticValue],
    controlArgs :: Maybe (ModuleControl Ident),
    inArgs :: [Either Constant Ident],
    outArgs :: NonEmpty Ident
  }
  deriving (Show)

data Expr
  = AssignExpr Ident (Either Constant Ident)
  | InstExpr ModuleInst
  deriving (Show)
        \end{minted}
    \end{small}

    Monads Used: 12 $(+ 0)$.
\end{frame}

\begin{frame}[fragile]{Verilog Output}
    \begin{small}
        \begin{minted}{verilog}
    module node_main (
        input wire clock,
        input wire init,
        input wire [7:0] var_x,
        input wire [7:0] var_y,
        output wire [7:0] var_res
    );
        lustre_and #(
            .N(8)
        ) call_lustre_and_1 (
            .lhs(var_x),
            .rhs(var_y),
            .res(var_res)
        );
    endmodule
        \end{minted}
    \end{small}

    Monads Used: 13 $(+ 1)$.
\end{frame}

\subsection{Standard Library}

\begin{frame}[fragile]{Standard Library Design}
    \begin{minipage}{0.49\textwidth}
        \begin{minted}[fontsize=\footnotesize]{verilog}
module lustre_and #(
    parameter N = 1
) (
    input  wire [N-1:0] lhs,
    input  wire [N-1:0] rhs,
    output wire [N-1:0] res
);
  and (res, lhs, rhs);
endmodule
        \end{minted}
    \end{minipage}
    \hfill
    \begin{minipage}{0.49\textwidth}
        \begin{minted}[fontsize=\footnotesize]{verilog}
module lustre_sub #(parameter N = 1)
    ( ... );

  wire [N-1:0] new_rhs;
  not (new_rhs, rhs);

  internal_lustre_adder #(.N(N))
    adder (
      .lhs(lhs), .rhs(new_rhs),
      .carry_in(1'd1), .res(res),
      .flag_Z(), .flag_N(),
      .flag_C(), .flag_V());
endmodule
        \end{minted}
    \end{minipage}
\end{frame}

\begin{frame}[fragile]{The Lustre ALU}
    \begin{footnotesize}
        \begin{minted}{verilog}
    module internal_lustre_adder #(parameter N = 1) ( ... );
        wire [N:0] carry;
        assign carry[0] = carry_in;

        genvar i;
        generate
        for (i = 0; i < N; i = i + 1) begin : full_adder_gen
            internal_lustre_bit_adder single_bit_adder (
                .a(lhs[i]), .b(rhs[i]), .carry_in(carry[i]),
                .res(res[i]), .carry_out(carry[i+1]));
        end
        endgenerate

        assign flag_Z = ~(|res);
        assign flag_N = res[N-1];
        assign flag_C = carry[N];
        xor (flag_V, carry[N], carry[N-1]);
    endmodule
        \end{minted}
    \end{footnotesize}
\end{frame}

\section{Results}

\begin{frame}{In Short}
    \begin{itemize}
        \setlength{\itemsep}{5pt}
        \item $\approx 2700$ lines of Haskell,
        \item $\approx 300$ lines of Verilog,
        \item $\approx 700$ lines of C++.
    \end{itemize}
\end{frame}

\begin{frame}{Possibles Improvements}
    \begin{itemize}
        \setlength{\itemsep}{5pt}
        \item Enlarge the Lustre core (\texttt{pre}, \texttt{->}, \texttt{merge}, \dots),
        \item Add custom data types and structures,
        \item Improve Verilog code generation,
        \item Add buffer registers to reduce the critical path,
        \item Define a Lustre processor!
    \end{itemize}
\end{frame}

\end{document}
